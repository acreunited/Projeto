%%________________________________________________________________________
%% LEIM | PROJETO
%% 2022 / 2013 / 2012
%% Modelo para relat�rio
%% v04: altera��o ADEETC para DEETC; outros ajustes
%% v03: corre��o de gralhas
%% v02: inclui anexo sobre utiliza��o sistema controlo de vers�es
%% v01: original
%% PTS / MAR.2022 / MAI.2013 / 23.MAI.2012 (constru�do)
%%________________________________________________________________________




%%________________________________________________________________________
\myPrefaceChapter{Resumo}
%%________________________________________________________________________


A reedi��o de um jogo antigo chamado \emph{Naruto-Arena} \cite{NarutoArena}, trouxe nostalgia a um n�mero consider�vel de pessoas. Tendo sido parte da inf�ncia de muitos jogadores durante v�rios anos, estes t�m demonstrado interesse por vers�es alternativas do seu jogo preferido.

Neste �mbito, de forma a satisfazer este mercado inexplorado, considera-se relevante a cria��o de um novo jogo que engloba diferentes animes, originando novas personagens, combina��es, miss�es e classifica��es.

O objetivo deste projeto, denominado \emph{Anime-Arena}, � conceber e implementar uma plataforma de jogos para a web, que seja capaz de gerir a intera��o entre pares de jogadores que competem para melhorar a sua classifica��o e desbloquear personagens.

Todos os jogadores possuem personagens por omiss�o, de forma a conseguirem jogar. De modo a entrar em jogo contra um oponente, o utilizador necessita escolher tr�s personagens diferentes e selecionar o modo de jogo que pretende. Vencer e perder afeta o n�vel do jogador (e por conseguinte, a sua classifica��o) e/ou o estado das miss�es. Estas, requerem um n�vel m�nimo para puderem ser realizadas e, quando conclu�das, devolvem ao jogador a personagem que lhe estava associada, permitindo-o jogar com ela.   

%%________________________________________________________________________
\myPrefaceChapter{Abstract}
%%________________________________________________________________________

Write here an overview of your work \ldots

Motivation, most relevant ideas, main contributions, evaluations and brief conclusions.

Short sentences. Succinct paragraphs. Top-down approach.