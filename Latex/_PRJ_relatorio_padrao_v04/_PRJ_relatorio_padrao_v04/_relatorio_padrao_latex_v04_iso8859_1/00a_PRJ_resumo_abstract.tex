%%________________________________________________________________________
%% LEIM | PROJETO
%% 2022 / 2013 / 2012
%% Modelo para relat�rio
%% v04: altera��o ADEETC para DEETC; outros ajustes
%% v03: corre��o de gralhas
%% v02: inclui anexo sobre utiliza��o sistema controlo de vers�es
%% v01: original
%% PTS / MAR.2022 / MAI.2013 / 23.MAI.2012 (constru�do)
%%________________________________________________________________________




%%________________________________________________________________________
\myPrefaceChapter{Resumo}
%%________________________________________________________________________


A reedi��o de um jogo antigo chamado \emph{Naruto-Arena}, trouxe nostalgia a um n�mero consider�vel de pessoas. Tendo sido parte da inf�ncia de muitos jogadores durante v�rios anos, estes t�m demonstrado interesse por vers�es alternativas do seu jogo preferido.

Neste �mbito, de forma a satisfazer este mercado inexplorado, considera-se relevante a cria��o de um jogo que englobe diferentes s�ries animadas japonesas (animes), introduzindo novas personagens, combina��es, miss�es e classifica��es.

O objetivo deste projeto, denominado \emph{Anime-Arena}, � conceber e implementar uma plataforma para a web, que seja capaz de gerir a intera��o entre pares de jogadores que competem para melhorar a sua classifica��o e desbloquear personagens.

Para solu��o da aplica��o, utilizou-se JSP de modo a criar p�ginas din�micas baseadas em HTML que possuem tamb�m Javascript, de forma a melhorar a experi�ncia do utilizador e enviar pedidos ao servidor (Servlets) com Ajax, sendo poss�vel ler e escrever na base de dados que foi constru�da em MySQL e XML.



%%________________________________________________________________________
\myPrefaceChapter{Abstract}
%%________________________________________________________________________

Write here an overview of your work \ldots

Motivation, most relevant ideas, main contributions, evaluations and brief conclusions.

Short sentences. Succinct paragraphs. Top-down approach.